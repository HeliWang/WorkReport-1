% uw-wkrpt-se.tex - An example work report that uses uw-wkrpt.cls
% Copyright (C) 2002,2003  Simon Law
% 
% This program is free software; you can redistribute it and/or modify
% it under the terms of the GNU General Public License as published by
% the Free Software Foundation; either version 2 of the License, or
% (at your option) any later version.
% 
% This program is distributed in the hope that it will be useful,
% but WITHOUT ANY WARRANTY; without even the implied warranty of
% MERCHANTABILITY or FITNESS FOR A PARTICULAR PURPOSE.  See the
% GNU General Public License for more details.
% 
% You should have received a copy of the GNU General Public License
% along with this program; if not, write to the Free Software
% Foundation, Inc., 59 Temple Place, Suite 330, Boston, MA  02111-1307  USA
%
%%%%%%%%%%%%%%%%%%%%%%%%%%%%%%%%%%%%%%%%%%%%%%%%%%%%%%%%%%%%%%%%%%%%%
%
% We begin by calling the workreport class which includes all the
% definitions for the macros we will use.
\documentclass[se]{uw-wkrpt}

% We will use some packages to add functionality
\usepackage{graphicx} % Include graphic importing

% Now we will begin writing the document.
\begin{document}

%%%%%%%%%%%%%%%%%%%%%%%%%%%%%%%%%%%%%%%%%%%%%%%%%%%%%%%%%%%%%%%%%%%%%
%% IMPORTANT INFORMATION
%%%%%%%%%%%%%%%%%%%%%%%%%%%%%%%%%%%%%%%%%%%%%%%%%%%%%%%%%%%%%%%%%%%%%


%% First we, should create a title page.  This is done below:
% Fill in the title of your report.
\title{Development of an iOS Application}

% Fill in your employer's name.
\employer{Crouton Labs Incorperated}

% Fill in your employer's city and province.
\employeraddress{Kitchener, ON}

% Fill in your term.
\term{2B}

% If you want to specify the date, fill it in here.  If you comment out
% this line, today's date will be substituted.
%\date{April 26, 2003}

% If you are writing a "Confidential 1" report, uncomment the next line.
%\confidential{Confidential-1}

%%%%%%%%%%%%%%%%%%%%%%%%%%%%%%%%%%%%%%%%%%%%%%%%%%%%%%%%%%%%%%%%%%%%%
%% Information that wont change
%%%%%%%%%%%%%%%%%%%%%%%%%%%%%%%%%%%%%%%%%%%%%%%%%%%%%%%%%%%%%%%%%%%%%

\author{Jake Nielsen}

\uwid{20338042}

\address{40 Karch st.,\\*
         Cambridge, ON\ \ N3C 1Y5}

\school{University of Waterloo}

\faculty{Mechanical and Mechatronics Engineering}

\email{jake.k.nielsen@gmail.com}

\program{Mechatronics Engineering}

\chair{Dr.\ Pearl\ Sullivan}

\chairaddress{Mechanical and Mechatronics Engineering Department,\\*
              University of Waterloo,\\*
	      Waterloo, ON\ \ N2L 3G1}



% Now, we ask LaTeX to generate the title.
\maketitle

%%%%%%%%%%%%%%%%%%%%%%%%%%%%%%%%%%%%%%%%%%%%%%%%%%%%%%%%%%%%%%%%%%%%%
%% FRONT MATTER
%%%%%%%%%%%%%%%%%%%%%%%%%%%%%%%%%%%%%%%%%%%%%%%%%%%%%%%%%%%%%%%%%%%%%
%% \frontmatter will make the \section commands ignore their numbering,
%% it will also use roman page numbers.
\frontmatter

% After this, we must create a letter of submission.
\begin{letter}
I have just completed my fourth work term, following my \theterm{} term.
Please find enclosed my first work term report entitled:
``\thetitle'' at \theemployer.  

I am one of the co-founders of Crouton Labs Inc. We make smartphone
apps for restaurants in the Kitchener/Waterloo area. In particular, this
report discusses the development choices that I made as the iOS developer
on our development team while designing and programming the app that we
made for Pita Factory.

I have had no direct assistance from anyone.  I do wish to thank Christophe
Biocca for his invaluable support in learning and working with the objective-c
language. I can say without a doubt that my knowledge of the language, its features,
and coding practices would not be half of what they are today if not for Christophe's
expertise.

% Note that I do not need to type out the boilerplate confirmation,
% nor do I need to write a signature block.  This is generated for me.
% We are now finished with the letter.
\end{letter}

% Next, we need to make a Table of Contents, List of Figures and 
% List of Tables.  You will most likely need to run LaTeX twice to
% get these correct.  The first pass for LaTeX to figure out the
% labels, and the second pass to put in the right references.
\tableofcontents
\listoffigures
\listoftables

\section{Summary}

INPUT SUMMARY HERE AFTER COMPLETING THE REST OF THE WORK REPORT



%%%%%%%%%%%%%%%%%%%%%%%%%%%%%%%%%%%%%%%%%%%%%%%%%%%%%%%%%%%%%%%%%%%%%
%% REPORT BODY
%%%%%%%%%%%%%%%%%%%%%%%%%%%%%%%%%%%%%%%%%%%%%%%%%%%%%%%%%%%%%%%%%%%%%
%% \main will make the \section commands numbered again,
%% it will also use arabic page numbers.
\mainmatter

% You must have an Introduction
\section{Introduction}\label{sec:intro}

The purpose of this report is to convey the challenges that were faced
during the development of the iOS application that Crouton Labs Inc. 
provides to Pita Factory.

In order to undestand the motivation for writing this app in the first
place, it's important to understand what problem this app is intended
to solve. This app is an attempt to reduce the time spent in fast food restaurants
waiting in line and processing transactions. This should allow for the restaurant
to service more customers and thus increase profits. With this in mind,
this main goals of this app are to allow a user to browse the restaurant's
menu, construct an order that they would like to purchase, and to provide
a means of processing the transaction through the app.

The programming language that was used to accomplish this goal is called
objective-c. Objective-c is syntactically similar to c and smalltalk. Its strange
syntactic combination can be confusing at first, as there are often two
syntactically different ways of achieving exactly the same effect. Objective-c
at first glance does not appear to be much different from c++ in terms
of language features, but upon further inspection it can be seen that 
objective-c is perhaps better compared to a dynamic language. While objective-c
does not fit the true definition of a dynamic language, it has many 
language features that are often associated with dynamic languages. 
Object and class methods can be called by progamatic string construction
of the name of the method, which is refered to as a "selector".
Similarly, subclasses can be refered to via programatic strings. With some work,
classes can even request the names of their subclasses, allowing for
some very powerful programming techniques that will be discussed later in
this report.

This report will discuss the general coding practices that were adhered to
during development, the challenges faced during development, and the solutions
that were implimented to deal with the challenges that arose.

The actual interface flow is not within the scope of this report.
The user interface was designed by a 3rd party designer and was merely
implimented and occaisionally exteneded upon by Crouton Labs.

\section{Coding Practices}

\subsection{Motivation}

The coding practices that were maintained 
during the development process are an extremely
important factor that affected the final product of this development
process. Proper coding 
practices are often dismissed or overlooked by many programmers to
the detriment of their productivity and positive outlook. Adhering to 
good practices make the painful parts of software development much less
frequent and much more bearable.

\subsection{Version Control}

Using version control is the single, most important practice any
good programmer should have. While the choice of version control 
software is hotly debated, I am partial to git. Git is a very
powerful source control system and is also very widely used.
Part of good source control use is having a remote repository.
For this project, www.github.com was used as our remote repository.

All version control software has the concept of small, incrimental
changes that, when put together, move the development of the code-base
forward. In git, these are refered to as commits. It's good practice
to make very small, bite-sized commits. In escence, each commit should
fix one bug, impliment one feature, or add one more logical piece to
the codebase. In this way, if certain features aren't working, or simply
aren't finished, they can be excluded from release builds without any
painful merge conflicts.

If a feature is not trivial to impliment, the feature should be
made on a separate "branch" soas to maintain the small size of commits
while still making sure that it is known that the group of commits are 
associated (and likely dependant on) one another. In this way, if a feature
needs to be included or discluded, this can be done simply by including, or 
discluding the entire branch.

\subsection{Efficientcy vs Simplicity}

This is another widely overlooked and undervalued concept that 
helps immensely during the development process. Many programmers
sacrifice simplicity in favor of efficiency much more often than
they should. While it's important under certain circumstances to
have efficient code, it can be hard to know where optimization would
do the most good, or even if it is necessary. Optimization should
only happen if it becomes a problem. Overly optimized code is indistinguishable
from properly optimized code at runtime, but at development time, 
overly optimized code is much less readable, and will cause pain
for any contributors, or even for the person who did the optimization
in the first place. Hence, optimize only when necessary.

\subsection{Refactoring}

It seems obvious, but refactoring is esscencial to maintaining a usable and agile
codebase. To emphasize the importance that needs to be put into refactoring, the 
codebase for this project has been rewritten little-by-little about 3.3 times.
That is, for every line of code in the codebase, there were 3.3 lines added and deleted.
The general mantra that describes good refactoring is, "refactor early, refactor often".

\subsection{Design Patterns}

Design patterns are widely used methods of solving particular problems that can
come up very frequently in programming problems. It's good to have a large arsenal
of design patterns to call on when developing software. By using patterns that
are widely known, your code will be more readable to someone who has never seen
your code before, but has experience with design patterns. Knowlege of design
patterns can drastically increase the speed at which you arrive at solutions.
This is an area that was not adhered to as rigidly as the other proper coding practices,
but it is something that would have added a lot of value. The aggressive refactoring
resulted in a codebase that makes proper use of design patterns, but it would
have been better to have arrived at them initially.

\section{Challenges and Solutions}

\subsection{Section Structure}

In this section, the challenges that were faced will be introduced and
explained. The solutions that were implimented to solve these problems will
also be discussed. It's worth noting that when the software was being developed, 
the approach that was taken involved itemizing all of the challenges that were 
expected to come up so that all of them could be taken into consideration before 
writing a single line of code. In general, it's good to catch the challenges and
make sure that strategies for dealing with them can be incorporated nicely into
the code base. The earlier a challenge is spotted, the faster and more effectively
it can be solved.

\subsection{Code Testing}

In order to be certain that a software solution is relatively bug-free and stable,
it's often good practice to write unit tests and system tests to make sure that
the software operates as expected. In the case of an iOS application, this becomes
a very difficult thing to do. Most of the bugs that occur in iOS applications are
visual or system-behavioural. Since there is no straightforward way to write system
tests for user interfaces, none were written in this development process. Extensive 
manual testing was done prior to releases to make sure that no breaking bugs were
introduced to the live codebase. The lack of unit tests or automated system tests
meant that all of the other coding practices discussed in the previous section became
extremely important. The manual testing also needed to be extremely thorough to make
absolutely sure that no bugs were pushed to the general public. While somewhat frowned
upon, this method proved to be successful, however without almost obsessive adherence
to proper coding practices it likely would not have worked very well. As it stands, only
one release had a bug that was considered to be breaking, and was a relatively subtle
behavioural bug. In terms of stability, to date, there has never been a crash in a
production build.

\subsection{Object Serialization}

In order to serialize an object to save it to non-volitile memory,
serialization and deserialization methods need to be implimented for each class that
needs to be saved. There are 18 different classes within the codebase that need to
be serializable. This quickly became tedious. In order to solve this problem, a class
was created that utilizes some of the more powerful language features of objective-c.
In particular, objective-c allows classes to programatically reference the values
and the names of their member variables. By utilizing that feature, the class (called AutomagicalCoder)  is able
to cycle through all of its own member variables and store serialize them by encoding them
with their own string name as the key. In this way, any class that subclasses AutomagicalCoder
needn't worry about how to serialize itself, and instead just inherently knows how.

\subsection{User Interface Tools}

There are a lot of similarity between facets of a user interface. Particularly in this
application, there was often need to have a way of displaying a multitude of different
types of objects as cells in a table. Similarly, the behaviour invoked when the user
taps on these object cells with his/her finger is often (although not always) the same.
Both of these problems lent themselves well to the dispatcher design pattern. 

In the dispatcher pattern,
when it is necessary to do a similar action to many different types of data, instead of coding
each action separately, the data is sent to a "dispatcher" that looks through the data that
it knows how to handle, and decides accordingly which special actions to take depending on
the data. If the data is not of a known type, depending on the dispatcher, it is either correct
to take some default action, or simply to throw an error to let the programmer know that 
an unexpected piece of data was passed to the dispatcher. Objective-c lends itself to the 
dispatcher method, because of the language feature that allows a class to look through it's
own subclasses without the subclasses explicitly making themselves known to the parent class.

Without delving any deeper, a solid working knowledge of Apple's API is required to understand
most, if not all of the following paragraphs. An abbreviated explaination of Apple's API can 
be found in Appendix A.

In the case of displaying cells, the parent class (CustomViewCell) looks through it's subclass
and calls the canDisplayData method to query them about the data that it has been presented with.
This is shown in the figure below.

(insert figure here)

In this way, all that is necessary to fully define and display a list view is a list of the data
that should be displayed in the list. By dispatching to CustomViewCell, the particularities of
displaying each individual cell needn't be handled in any way by the tableViewDelegate. 

Another language feature involved in the simplification of the TableView interface provided
by Apple is that of calling methods by constructing their name through string manipulation.
If there's somthing that needs to occur when the user clicks on a cell, the TableViewDelegate
need only impliment a function named <CELLNAME>Handler. The method (if it exists) will be called
by the table view delegate whenever the cell is clicked on. In this way, the delegate can 
distinguish between cell clicks without needing to actually know the type of the cell or the
position of the cell (which is the traditional way that Apple intends for cell distinctions to
be made). 

By making all of these simplifying abstractions, the code length of the files defining
the behaviour of particular pages is vastly. One class in particular (OrderComboViewController)
has 4 lines of code in its header file and 23 lines of code in its implimentation. A screenshot
of this particular page is shown below. 

(figure)

The UITableViewDataSource for this page is simplified in a similar way. The class responsible
for the page contents (OrderComboRenderer) has 7 lines in its header file, and 37 in its implimentation.


The dispatcher pattern is a very powerful way of simplifying code.

\subsection{Client Version Management}

One of the complicating factors in designing any application is how various versions of the code will
interact with one another. The two possible modes of interaction in this case are communication with
the server, and loading data cached by an older version of the client code. 

\subsubsection{Server Communication}

If an old version of the client code were to try to interact with the server, the server's behavior
may have to be different than if an up to date client app were to attempt communications. The way that
this is solved is again by dispatching. As part of every communication with the server, the iOS application
transmits its version number. Based on the version number, the server dispatches the the response behaviour
to the appropriate version of the server code.

\subsubsection{Client Data Caching}

A more difficult complication is that of loading cached data. If the ipod/iphone user has an old version of the
app, saves some data, updates their app to the current version, and then attempts to load the cached data,
bad things can happen. Depending on the version of the app that was used to cache the data, the procedure
to update the data to be consistant with the current model is different. To solve this problem, a combination
of the dispatcher pattern and the fallthrough pattern was used. The first step in determining what to do
with the data being loaded is to check the version string. Luckillty, this problem was anticipated, and 
the cached data always contains a version string. Because this all needs to be done by AutomagicalCoder,
no actual knowledge of what these variables are is known explicitly at load time. To solve this, Automagical
Coder checks the version string, and if the version string is not the current version string, it checks for
the existance of a recovery routine for each of the variables that it determines that it needs to load.
if a recovery routine is implimented in the child class, AutomagicalCoder dispatches to the recovery 
routine by calling <VARIABLENAME>Recovery. The recovery routine uses the fallthrough pattern.

The fallthrough pattern uses a switch case structure where the case clauses do not contain a break statement.
In this way, the variable information will be updated in a cascade through all of the versions starting
at the version that was found on the disc all the way up to the present model. In this way, the only
thing that needs to be updated from version to version is the logic for moving the data from the version
immediately prior to the current version. Everything else just cascades through.

\section{Conclusion}

In conclusion, by using the habits and patterns discussed in this report, the development project was a 
great success. From a technological view point, this product turned out great. It's stable, it looks good,
it is optimized enough that it feels smooth to interact with, and its code is very readable. 

The app can currently be downloaded from the app store, and is called "Pita Factory". As
not everyone has an iOS device, below are some screenshots of the finished product.

(Screenshots here)

%% example of a figure
\begin{figure}
  \centering
  \begin{picture}(300,170)
    % Draw the text boxes
    \put( 50, 150){\makebox(0,0){\fbox{\texttt{document.tex}}}}
    \put(250, 150){\makebox(0,0){\fbox{\texttt{document.bib}}}}
    \put( 50, 100){\makebox(0,0)[l]{\fbox{\texttt{document.pdf}}}}
    \put(180, 100){\makebox(0,0){\fbox{\texttt{document.aux}}}}
    \put(250,  50){\makebox(0,0){\fbox{\texttt{document.bbl}}}}
    \put(150,   0){\makebox(0,0){\fbox{\texttt{document.pdf}}}}
    % Draw the connecting lines
    \put( 75, 143){\vector( 0,-1){ 35}} % .tex -> .pdf
    \put( 75, 125){\line  ( 1, 0){105}} % .tex -> .aux
    \put(180, 125){\vector( 0,-1){ 18}} % .tex -> .aux
    \put(205,  93){\line  ( 0,-1){ 18}} % .aux -- .bbl
    \put(205,  75){\line  ( 1, 0){ 45}} % .aux -- .bbl
    \put(250, 143){\vector( 0,-1){ 86}} % .bib -> .bbl
    \put( 25, 143){\line  ( 0,-1){118}} % .tex -- .pdf
    \put(155,  93){\line  ( 0,-1){ 68}} % .aux -- .pdf
    \put(250,  43){\line  ( 0,-1){ 18}} % .bbl -- .pdf
    \put( 25,  25){\line  ( 1, 0){225}} % ------------
    \put(150,  25){\vector( 0,-1){ 18}} % .*   -> .pdf
    % Draw the text
    \put( 78, 138){\makebox(0,0)[tl]{\textsc{pdf}\LaTeX{}}}
    \put(247,  60){\makebox(0,0)[br]{\BibTeX{}}}
    \put(153,  10){\makebox(0,0)[bl]{\textsc{pdf}\LaTeX{}}}
  \end{picture}
  \caption{Control flow of a \LaTeX{} compilation.}
  \label{fig:flow}
\end{figure}

%% Don't forget to refer to all your appendices in the main report.
\appendix

\section{Apple API Abbreviated}

Appendix A is an abbreviated description of Apples's iOS API. It
will explain the general concepts behind displaying things on
an iOS device as well as the specific elements of the API that
are pertainant to this report.

As a general overview, there are views, models, and controllers.
The views represent the actual images that the user sees. The models
represent the data that is meant to be conveyed to the user. The
controllers encapsulate the ways that the user can interact with
the data. This is known as the model-view-controller design pattern
and is the design pattern that apple intends for programmers to use
when interacting with their API.

\subsection{UIViews}

UIView is the most basic class that is meant to be displayed to
the screen. A UIView has an area of the screen that it can manipulate
via its drawrect method. A UIView can also have an arbitrary number of 
subviews which draw themselves onto the parent UIView. There are 
many built-in subclasses of UIView that can accomodate most 
applications that a person might want to build.

Some notable subclasses of UIView are: UILabel, UIButton, UITableView
UIViewCell, and UIImageView.

\subsubsection{UITableView}

A UITableView is a view that can display any number of scrollable 
UIViewCells. The UITableView must be associated with a UITableViewDelegate
in order to define any interactive behaviour and must be associated
with a UITableViewDataSource in order to have any content. they will be
discussed more in the UIViewControllers and UIDataSources sections.

\subsubsection{UIViewCell}

A UIViewCell is a UIView that can be produced by a UITableViewDataSource.
The vanila UIViewCell can be used by itself as elements to populate a
UITableView and can be configured in several different default configurations
that can accomodate most data that a person might want to display reasonably
well. In cases where the vanila UIViewCell will not suffice, UIViewCell can
be subclassed to accomodate arbitrarily complex and unusual cells.

\subsection{UIViewControllers}

UIViewController is a class that is not intended to be used by itself, 
but rather to be subclassed to accomodate specific behaviour that
a programmer might intend for a view to have. View controllers are
often made to be the delegates of the view that they control so that
a view can request behavior from its view controller without actually
needing to know anything specific about the view controller. 

\subsubsection{UITableViewDelegate}

The most notable type of delegation that this report refers to is
that of the UITableViewDelegate. The UITableViewDelegate is responsible
for telling the UITableView what needs to happen upon the selection of
a cell at a particular position on the table. It's also responsible for 
letting the UITableView know the height that each cell is intended to
occupy.

\subsection{UIDataSources}

In general, the "model" part of the model-view-controller pattern is 
satisfied by a class that impliments 

Currently, there are some known problems with this document class.
\begin{itemize}
  \item It is not officially supported or acknowledged by the
        E\&CE department.
  \item Not all users have converted to using a typesetting language, and
        insist on using word processors.
  \item It does not bring world peace.
\end{itemize}

Fixes for these bugs are most certainly welcome.  Please provide a patch
against the document class document.

\section{Colophon}\label{app:colophon}
This sample document was written by Simon Law, a third-year Computer
Engineering student at the University of Waterloo, in Waterloo, ON, CA.
When he is not programming, he can be found reading or sleeping;
both of which are his favourite activities.\footnote{OK, so I don't
have a life yet.  I'm working on it.}

The best way to contact him is by e-mail, at \url{sfllaw@uwaterloo.ca}.

This document was implemented using the \texttt{ece} variant of the
\texttt{uw-wkrpt} document class.  The document class, and the 
surrounding documentation is implemented using the \LaTeXe{} macro 
package which is built on the \TeX{} typesetting system.  The documents
were generated by the web2c implementation of \TeX, found in the 
\teTeX{} distribution.  The typeface used is Computer Modern.

The entire system was written in the Vim text editor. The operating
system used was Debian GNU/Linux which ran on an IBM ThinkPad A20m. This
stalwart companion allowed him to work on this report periodically, even
during his ``off'' time up at the cottage.


\section{GNU General Public License}\label{app:gnugpl}
\renewcommand{\labelenumii}{\theenumii)}
\begin{singlespacing}
\centerline{Version 2, June 1991}

\begin{quote}
 Copyright \copyright{} 1989, 1991 Free Software Foundation, Inc.\newline
 \null\hspace{0.5in}59 Temple Place, Suite 330, Boston, 
 MA\ \ 02111-1307\ \ USA\\
 Everyone is permitted to copy and distribute verbatim copies
 of this license document, but changing it is not allowed.
\end{quote}

\subsection*{\centerline{Preamble}}

  The licenses for most software are designed to take away your
freedom to share and change it.  By contrast, the GNU General Public
License is intended to guarantee your freedom to share and change free
software--to make sure the software is free for all its users.  This
General Public License applies to most of the Free Software
Foundation's software and to any other program whose authors commit to
using it.  (Some other Free Software Foundation software is covered by
the GNU Library General Public License instead.)  You can apply it to
your programs, too.

  When we speak of free software, we are referring to freedom, not
price.  Our General Public Licenses are designed to make sure that you
have the freedom to distribute copies of free software (and charge for
this service if you wish), that you receive source code or can get it
if you want it, that you can change the software or use pieces of it
in new free programs; and that you know you can do these things.

  To protect your rights, we need to make restrictions that forbid
anyone to deny you these rights or to ask you to surrender the rights.
These restrictions translate to certain responsibilities for you if you
distribute copies of the software, or if you modify it.

  For example, if you distribute copies of such a program, whether
gratis or for a fee, you must give the recipients all the rights that
you have.  You must make sure that they, too, receive or can get the
source code.  And you must show them these terms so they know their
rights.

  We protect your rights with two steps: (1) copyright the software, and
(2) offer you this license which gives you legal permission to copy,
distribute and/or modify the software.

  Also, for each author's protection and ours, we want to make certain
that everyone understands that there is no warranty for this free
software.  If the software is modified by someone else and passed on, we
want its recipients to know that what they have is not the original, so
that any problems introduced by others will not reflect on the original
authors' reputations.

  Finally, any free program is threatened constantly by software
patents.  We wish to avoid the danger that redistributors of a free
program will individually obtain patent licenses, in effect making the
program proprietary.  To prevent this, we have made it clear that any
patent must be licensed for everyone's free use or not licensed at all.

  The precise terms and conditions for copying, distribution and
modification follow.

\subsection*{\centerline{TERMS AND CONDITIONS FOR COPYING,}\\
             \centerline{DISTRIBUTION AND MODIFICATION}}

\begin{enumerate}
\setcounter{enumi}{-1}

\item This License applies to any program or other work which contains
a notice placed by the copyright holder saying it may be distributed
under the terms of this General Public License.  The ``Program'', below,
refers to any such program or work, and a ``work based on the Program''
means either the Program or any derivative work under copyright law:
that is to say, a work containing the Program or a portion of it,
either verbatim or with modifications and/or translated into another
language.  (Hereinafter, translation is included without limitation in
the term ``modification''.)  Each licensee is addressed as ``you''.

Activities other than copying, distribution and modification are not
covered by this License; they are outside its scope.  The act of
running the Program is not restricted, and the output from the Program
is covered only if its contents constitute a work based on the
Program (independent of having been made by running the Program).
Whether that is true depends on what the Program does.

\item You may copy and distribute verbatim copies of the Program's
source code as you receive it, in any medium, provided that you
conspicuously and appropriately publish on each copy an appropriate
copyright notice and disclaimer of warranty; keep intact all the
notices that refer to this License and to the absence of any warranty;
and give any other recipients of the Program a copy of this License
along with the Program.

You may charge a fee for the physical act of transferring a copy, and
you may at your option offer warranty protection in exchange for a fee.

\item You may modify your copy or copies of the Program or any portion
of it, thus forming a work based on the Program, and copy and
distribute such modifications or work under the terms of Section 1
above, provided that you also meet all of these conditions:

  \begin{enumerate}
    \item You must cause the modified files to carry prominent notices
    stating that you changed the files and the date of any change.

    \item You must cause any work that you distribute or publish, that in
    whole or in part contains or is derived from the Program or any
    part thereof, to be licensed as a whole at no charge to all third
    parties under the terms of this License.

    \item If the modified program normally reads commands interactively
    when run, you must cause it, when started running for such
    interactive use in the most ordinary way, to print or display an
    announcement including an appropriate copyright notice and a
    notice that there is no warranty (or else, saying that you provide
    a warranty) and that users may redistribute the program under
    these conditions, and telling the user how to view a copy of this
    License.  (Exception: if the Program itself is interactive but
    does not normally print such an announcement, your work based on
    the Program is not required to print an announcement.)
  \end{enumerate}

These requirements apply to the modified work as a whole.  If
identifiable sections of that work are not derived from the Program,
and can be reasonably considered independent and separate works in
themselves, then this License, and its terms, do not apply to those
sections when you distribute them as separate works.  But when you
distribute the same sections as part of a whole which is a work based
on the Program, the distribution of the whole must be on the terms of
this License, whose permissions for other licensees extend to the
entire whole, and thus to each and every part regardless of who wrote it.

Thus, it is not the intent of this section to claim rights or contest
your rights to work written entirely by you; rather, the intent is to
exercise the right to control the distribution of derivative or
collective works based on the Program.

In addition, mere aggregation of another work not based on the Program
with the Program (or with a work based on the Program) on a volume of
a storage or distribution medium does not bring the other work under
the scope of this License.

\item You may copy and distribute the Program (or a work based on it,
under Section 2) in object code or executable form under the terms of
Sections 1 and 2 above provided that you also do one of the following:

    \begin{enumerate}
    \item Accompany it with the complete corresponding machine-readable
    source code, which must be distributed under the terms of Sections
    1 and 2 above on a medium customarily used for software interchange; or,

    \item Accompany it with a written offer, valid for at least three
    years, to give any third party, for a charge no more than your
    cost of physically performing source distribution, a complete
    machine-readable copy of the corresponding source code, to be
    distributed under the terms of Sections 1 and 2 above on a medium
    customarily used for software interchange; or,

    \item Accompany it with the information you received as to the offer
    to distribute corresponding source code.  (This alternative is
    allowed only for noncommercial distribution and only if you
    received the program in object code or executable form with such
    an offer, in accord with Subsection b above.)
    \end{enumerate}

The source code for a work means the preferred form of the work for
making modifications to it.  For an executable work, complete source
code means all the source code for all modules it contains, plus any
associated interface definition files, plus the scripts used to
control compilation and installation of the executable.  However, as a
special exception, the source code distributed need not include
anything that is normally distributed (in either source or binary
form) with the major components (compiler, kernel, and so on) of the
operating system on which the executable runs, unless that component
itself accompanies the executable.

If distribution of executable or object code is made by offering
access to copy from a designated place, then offering equivalent
access to copy the source code from the same place counts as
distribution of the source code, even though third parties are not
compelled to copy the source along with the object code.

\item You may not copy, modify, sublicense, or distribute the Program
except as expressly provided under this License.  Any attempt
otherwise to copy, modify, sublicense or distribute the Program is
void, and will automatically terminate your rights under this License.
However, parties who have received copies, or rights, from you under
this License will not have their licenses terminated so long as such
parties remain in full compliance.

\item You are not required to accept this License, since you have not
signed it.  However, nothing else grants you permission to modify or
distribute the Program or its derivative works.  These actions are
prohibited by law if you do not accept this License.  Therefore, by
modifying or distributing the Program (or any work based on the
Program), you indicate your acceptance of this License to do so, and
all its terms and conditions for copying, distributing or modifying
the Program or works based on it.

\item Each time you redistribute the Program (or any work based on the
Program), the recipient automatically receives a license from the
original licensor to copy, distribute or modify the Program subject to
these terms and conditions.  You may not impose any further
restrictions on the recipients' exercise of the rights granted herein.
You are not responsible for enforcing compliance by third parties to
this License.

\item If, as a consequence of a court judgment or allegation of patent
infringement or for any other reason (not limited to patent issues),
conditions are imposed on you (whether by court order, agreement or
otherwise) that contradict the conditions of this License, they do not
excuse you from the conditions of this License.  If you cannot
distribute so as to satisfy simultaneously your obligations under this
License and any other pertinent obligations, then as a consequence you
may not distribute the Program at all.  For example, if a patent
license would not permit royalty-free redistribution of the Program by
all those who receive copies directly or indirectly through you, then
the only way you could satisfy both it and this License would be to
refrain entirely from distribution of the Program.

If any portion of this section is held invalid or unenforceable under
any particular circumstance, the balance of the section is intended to
apply and the section as a whole is intended to apply in other
circumstances.

It is not the purpose of this section to induce you to infringe any
patents or other property right claims or to contest validity of any
such claims; this section has the sole purpose of protecting the
integrity of the free software distribution system, which is
implemented by public license practices.  Many people have made
generous contributions to the wide range of software distributed
through that system in reliance on consistent application of that
system; it is up to the author/donor to decide if he or she is willing
to distribute software through any other system and a licensee cannot
impose that choice.

This section is intended to make thoroughly clear what is believed to
be a consequence of the rest of this License.

\item If the distribution and/or use of the Program is restricted in
certain countries either by patents or by copyrighted interfaces, the
original copyright holder who places the Program under this License
may add an explicit geographical distribution limitation excluding
those countries, so that distribution is permitted only in or among
countries not thus excluded.  In such case, this License incorporates
the limitation as if written in the body of this License.

\item The Free Software Foundation may publish revised and/or new versions
of the General Public License from time to time.  Such new versions will
be similar in spirit to the present version, but may differ in detail to
address new problems or concerns.

Each version is given a distinguishing version number.  If the Program
specifies a version number of this License which applies to it and ``any
later version'', you have the option of following the terms and conditions
either of that version or of any later version published by the Free
Software Foundation.  If the Program does not specify a version number of
this License, you may choose any version ever published by the Free Software
Foundation.

\item If you wish to incorporate parts of the Program into other free
programs whose distribution conditions are different, write to the author
to ask for permission.  For software which is copyrighted by the Free
Software Foundation, write to the Free Software Foundation; we sometimes
make exceptions for this.  Our decision will be guided by the two goals
of preserving the free status of all derivatives of our free software and
of promoting the sharing and reuse of software generally.
\newcounter{Enumi}
\setcounter{Enumi}{\value{enumi}}
\end{enumerate}

\subsubsection*{\centerline{NO WARRANTY}}

\begin{enumerate}
\setcounter{enumi}{\value{Enumi}}
\item BECAUSE THE PROGRAM IS LICENSED FREE OF CHARGE, THERE IS NO WARRANTY
FOR THE PROGRAM, TO THE EXTENT PERMITTED BY APPLICABLE LAW.  EXCEPT WHEN
OTHERWISE STATED IN WRITING THE COPYRIGHT HOLDERS AND/OR OTHER PARTIES
PROVIDE THE PROGRAM ``AS IS'' WITHOUT WARRANTY OF ANY KIND, EITHER EXPRESSED
OR IMPLIED, INCLUDING, BUT NOT LIMITED TO, THE IMPLIED WARRANTIES OF
MERCHANTABILITY AND FITNESS FOR A PARTICULAR PURPOSE.  THE ENTIRE RISK AS
TO THE QUALITY AND PERFORMANCE OF THE PROGRAM IS WITH YOU.  SHOULD THE
PROGRAM PROVE DEFECTIVE, YOU ASSUME THE COST OF ALL NECESSARY SERVICING,
REPAIR OR CORRECTION.

\item IN NO EVENT UNLESS REQUIRED BY APPLICABLE LAW OR AGREED TO IN WRITING
WILL ANY COPYRIGHT HOLDER, OR ANY OTHER PARTY WHO MAY MODIFY AND/OR
REDISTRIBUTE THE PROGRAM AS PERMITTED ABOVE, BE LIABLE TO YOU FOR DAMAGES,
INCLUDING ANY GENERAL, SPECIAL, INCIDENTAL OR CONSEQUENTIAL DAMAGES ARISING
OUT OF THE USE OR INABILITY TO USE THE PROGRAM (INCLUDING BUT NOT LIMITED
TO LOSS OF DATA OR DATA BEING RENDERED INACCURATE OR LOSSES SUSTAINED BY
YOU OR THIRD PARTIES OR A FAILURE OF THE PROGRAM TO OPERATE WITH ANY OTHER
PROGRAMS), EVEN IF SUCH HOLDER OR OTHER PARTY HAS BEEN ADVISED OF THE
POSSIBILITY OF SUCH DAMAGES.
\end{enumerate}

\subsection*{\centerline{END OF TERMS AND CONDITIONS}}

\subsection*{\centerline{How to Apply These Terms to Your New Programs}}

  If you develop a new program, and you want it to be of the greatest
possible use to the public, the best way to achieve this is to make it
free software which everyone can redistribute and change under these terms.

  To do so, attach the following notices to the program.  It is safest
to attach them to the start of each source file to most effectively
convey the exclusion of warranty; and each file should have at least
the ``copyright'' line and a pointer to where the full notice is found.
% This is a hack to make the quote environment behave nicely.
\renewenvironment{quote}{\list{}{}\item\relax}{\endlist}
\begin{quote}\ttfamily\footnotesize
    \emph{one line to give the program's name and a brief 
    idea of what it\nolinebreak[4] does.}\\
    Copyright (C) \emph{year}\ \ \emph{name of author}\\
\mbox{}\\
    This program is free software; you can redistribute it and/or modify\\
    it under the terms of the GNU General Public License as published by\\
    the Free Software Foundation; either version 2 of the License, or\\
    (at your option) any later version.\\
\mbox{}\\
    This program is distributed in the hope that it will be useful,\\
    but WITHOUT ANY WARRANTY; without even the implied warranty of\\
    MERCHANTABILITY or FITNESS FOR A PARTICULAR PURPOSE.  See the\\
    GNU General Public License for more details.\\
\mbox{}\\
    You should have received a copy of the GNU General Public License\\
    along with this program; if not, write to the Free Software\\
    Foundation, Inc., 59 Temple Place, Suite 330, Boston, \\
    MA\ \ 02111-1307\ \ USA
\end{quote}


Also add information on how to contact you by electronic and paper mail.

If the program is interactive, make it output a short notice like this
when it starts in an interactive mode:
\begin{quote}\ttfamily\footnotesize
    Gnomovision version 69, Copyright (C) \emph{year}\ \ 
    \emph{name of author}\\
    Gnomovision comes with ABSOLUTELY NO WARRANTY; for details type `show w'.\\
    This is free software, and you are welcome to redistribute it\\
    under certain conditions; type `show c' for details.
\end{quote}

The hypothetical commands `show w' and `show c' should show the appropriate
parts of the General Public License.  Of course, the commands you use may
be called something other than `show w' and `show c'; they could even be
mouse-clicks or menu items--whatever suits your program.

You should also get your employer (if you work as a programmer) or your
school, if any, to sign a ``copyright disclaimer'' for the program, if
necessary.  Here is a sample; alter the names:
\begin{quote}\ttfamily\footnotesize
  Yoyodyne, Inc., hereby disclaims all copyright interest in the program\\
  `Gnomovision' (which makes passes at compilers) written by James Hacker.\\
\mbox{}\\
  \emph{signature of Ty Coon}, 1 April 1989\\
  Ty Coon, President of Vice
\end{quote}

This General Public License does not permit incorporating your program into
proprietary programs.  If your program is a subroutine library, you may
consider it more useful to permit linking proprietary applications with the
library.  If this is what you want to do, use the GNU Library General
Public License instead of this License.

\end{singlespacing}

\end{document}
